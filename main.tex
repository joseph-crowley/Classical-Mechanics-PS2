\documentclass{article}
\usepackage[utf8]{inputenc}
\usepackage[margin=1.0in]{geometry}
\usepackage{amsmath}
\usepackage{amssymb}
\usepackage{fancyhdr}
\usepackage{physics}
\usepackage{wrapfig}
\usepackage{hyperref}
\usepackage{multirow}
\usepackage{amsthm}



\renewcommand{\thesubsection}{\thesection\Alph{subsection}}
\renewcommand\qedsymbol{\square}



\title{Classical Mechanics PS1}
\author{Joe Crowley}
\date{October 2020}

\pagestyle{fancy}
\renewcommand{\headrulewidth}{0pt}
\renewcommand{\footrulewidth}{1pt}

\fancyhf{}
\rhead{
Joe Crowley \\
Physics 205 \\
Problem Set 2\\
}
\rfoot{Page \thepage}

\begin{document}  

\section{GPS Derivation 2.3}
\textit{Prove that the shortest distance between two points in space is a straight line.}
\begin{proof}
    The length of an infinitesimal arc along the path $ds$ is given by 
    \begin{align*}
        ds &= \sqrt{dx^2 + dy^2} \\ 
        &= \sqrt{1+ \left( \frac{dy}{dx}\right)^2 }dx,
    \end{align*}
    which can be integrated to find the total length of the path. Applying a variation to the length of the curve and setting it stationary yields 
    \begin{align*}
        S &= \int{ds}\\
        \delta S &= \int{\delta \sqrt{1+ \left( \frac{dy}{dx}\right)^2 }dx}\\
        &= \int{\frac{\frac{dy}{dx}} {\sqrt{1+ \left( \frac{dy}{dx}\right)^2 }} \delta \frac{dy}{dx}dx.
    \end{align*}
    Using the equivalence of mixed partials and integrating by parts, 
    \begin{align*}
        
    \end{align*}
\end{proof}

\end{document}
